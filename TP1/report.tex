\documentclass[a4paper,10pt]{article}

\usepackage[utf8]{inputenc}
\usepackage{t1enc}

\usepackage[utf8]{inputenc}
\usepackage{t1enc}
\usepackage[spanish]{babel}
\usepackage[pdftex,usenames,dvipsnames]{color}
\usepackage[pdftex]{graphicx}
\usepackage{enumerate}
\usepackage{amsmath}
\usepackage{amsfonts}
\usepackage{amssymb}
\usepackage[table]{xcolor}
\usepackage[small,bf]{caption}
\usepackage{float}
\usepackage{subfig}
\usepackage{listings}
\usepackage{bm}
\usepackage{times}
\usepackage{verbatim}
\usepackage{moreverb}
\usepackage{fancyvrb}
\usepackage{hyperref}
\usepackage{multirow}


\begin{document}
\setcounter{secnumdepth}{5}
\setcounter{tocdepth}{5}

\begin{titlepage}
        \vfill
        \thispagestyle{empty}
        \begin{center}
                \includegraphics{./images/itba_logo.png}
                \vfill
                \Huge{Criptografia y Seguridad}\\
                \vspace{1cm}
                \Huge{25 Errores de Software Mas Frecuentes}\\
                \vspace{1cm}
                \Huge{Trabajo Pr\'actico Especial 1}\\
        \end{center}
        \vfill
        \large{
        \begin{tabular}{lcr}
                Civile, Juan Pablo && 50453\\
                Crespo, Alvaro && 50758 \\
                Susnisky, Dario && \\
                Wassington, Axel && \\
        \end{tabular}
}
        \vspace{2cm}
        \begin{center}
                \large{22 de mayo de 2013}\\
        \end{center}
\end{titlepage}
\newpage

\setcounter{page}{1}

\section{Introducción}
La lista de los 25 errores de software más frecuentes es un proyecto impulsado por \textbf{SANS Institute}, \textbf{MITRE} y expertos en seguridad informática de EEUU 
y Europa. Se deriva de la experiencia del desarrollo del Top20\footnote{\href{http://www.sans.org/top20/}{http://www.sans.org/top20/}} de ataques informáticos
implementado por SANS, y la enumeración de debilidades comunes\footnote{\href{http://cwe.mitre.org/}{http://cwe.mitre.org/}} de software 
desarrollado por MITRE. Este \emph{ranking} se actualiza todos los años, y para la selección de los errores en cada año,
se toman en cuenta evaluaciones realizadas por distintas empresas del rubro de seguridad informática, las cuales evalúan a
los errores según \textbf{prevalecencia}, \textbf{importancia} y \textbf{posibilidad de \emph{exploit}}. Luego, todas estas
evaluaciones son las entradas para el sistema de puntaje diseñado por MITRE (\emph{Common Weakness Scoring System}), de donde se obtienen los
errores ordenados.\\
 
Los 25 errores están categorizados en los siguientes 3 grupos
\begin{itemize}
    \item Interacción insegura entre componentes.
    \item Gestión riesgosa de recursos.
    \item Defensas porosas.
\end{itemize}
 
En este año, fueron 20 las organizaciones evaluadoras, y el resultado de la evaluación arrojó el \emph{ranking} de errores que se muestra
en la siguiente sección.
 
\subsection{Los 25 errores más frecuentes del 2011}
Se presentan resaltados los errores que se analizan en este documento.
\begin{enumerate}
    \item SQL Injection
    \item OS Command Injection
    \item \textbf{Classic Buffer Overflow}
    \item Improper Neutralization of Input During Web Page Generation (``Cross-site Scripting'')
    \item Missing Authentication for Critical Function
    \item Missing Authorization
    \item Uso de credenciales hard-codeadas.
    \item \textbf{Missing Encryption of Sensitive Data}
    \item Unrestricted Upload of File with Dangerous Type
    \item Reliance on Untrusted Inputs in a Security Decision
    \item Execution with Unnecessary Privileges
    \item Cross-Site Request Forgery (CSRF)
    \item \textbf{Improper Limitation of a Pathname to a Restricted Directory ('Path Traversal')}
    \item Download of Code Without Integrity Check
    \item Incorrect Authorization
    \item Inclusion of Functionality from Untrusted Control Sphere
    \item Incorrect Permission Assignment for Critical Resource
    \item \textbf{Use of Potentially Dangerous Functionality}
    \item Use of a Broken or Risky Cryptographic Algorithm
    \item Incorrect Calculation of Buffer Size
    \item Improper Restriction of Excessive Authentication Attempts
    \item URL Redirection to Untrusted Site ('Open Redirect')
    \item \textbf{Uncontrolled Format String}
    \item Integer Overflow or Wraparound
    \item Use of a One-Way Hash without a Salt
\end{enumerate}

\subsection{Otras vulnerabilidades}
Otras vulnerabilidades que no fueron selecccionados para el \emph{Top25} son:

\begin{enumerate}
 \setcounter{enumi}{26}
 \item \textbf{Improper Validation of Array Index}
 \setcounter{enumi}{38}
 \item \textbf{Information Exposure Through an Error Message}
\end{enumerate}


\section{Errores}

\subsection{Buffer Copy without Checking Size of Input ('Classic Buffer Overflow')}

\subsubsection{Descripción del error}

Esta vulnerabilidad ocurre cuando un programa copia un buffer de entrada a otro de salida, sin verificar que el tamaño del buffer de entrada sea menor que el tamaño del buffer de salida. 
Esto es lo que se conoce como \textbf{buffer overflow}.\\

Una condición de buffer overflow existe cuando un programa intenta poner más datos en un buffer de los que éste puede contener, o cuando intenta poner datos en un área de memoria
fuera de los límites del buffer. El más simple de los errores, y la causa más común de los buffer overflows, es el típico caso en el que un programa copia un buffer sin
restringir cuanto se está copiando efectivamente. Existen otros casos, pero la existencia de un overflow clásico sugiere fuertemente que el programador no está considerando
las más básicas reglas de seguridad. \\

Previamente se conocía esta vulnerabilidad con el nombre de \textbf{Unbounded Transfer ('Classic Buffer Overflow')}. El 14 de Octubre de 2008 se le cambió el nombre por el actual.

\subsubsection{Terminología}
Términos alternativos
\begin{itemize}
    \item \textbf{Buffer Overrun}
    \item \textbf{Unbounded Transfer}
\end{itemize}

Muchos casos que son llamados ``buffer overflows'' son susbstancialmente diferentes al overflow ``clásico'', incluyendo diferentes tipos de bugs que se basan en técnicas
de overflow, como errores de signos de enteros, integer overflows y bugs de formatos de strings. Esta terminología imprecisa puede hacer que sea difícil determinar a cual
variante se este refiriendo.

\subsubsection{Detalles técnicos del error}
\begin{tabular}[\baselineskip]{|l|p{7cm}|}
  \hline
  \textbf{Categoría} & Manejo Riesgoso de Recursos \\
  \hline
  \textbf{Plataforma} & Lenguajes de Programación \\
  \hline
  \textbf{Tiempo de Introducción} & Implementación \\
  \hline
  \textbf{Lenguaje} & C, C++, Assembly \\
  \hline
  \textbf{Probabilidad de \emph{exploit}} & Alta a Muy Alta \\
  \hline
\end{tabular}

\subsubsection{Ejemplos de código}

\noindent \textbf{Ejemplo en C}\\

El siguiente código, escrito en C, le pide al usuario que ingrese su apellido y luego intenta guardar el valor ingresado en el array \texttt{last\_name}.

\begin{lstlisting}[frame=single]
    char last_name[20];
    printf ("Enter your last name: ");
    scanf ("%s", last_name);
\end{lstlisting}

El problema con este código es que no restringe o limita el tamaño del nombre ingresado por el usuario. Si el usuario ingresa "Very\_very\_long\_last\_name", que tiene 24 carácteres de largo,
entonces ocurrirá un buffer overflow, ya que el array solo puede contener 20 carácteres en total. \\

\noindent \textbf{Ejemplo en C}\\

El siguiente ejemplo, también un fragmento de código en C, intenta crear un copia local de un buffer para hacer alguna manipulación de los datos.

\begin{lstlisting}[frame=single]
    void manipulate_string(char* string){
    char buf[24];
    strcpy(buf, string);
    ...
    }
\end{lstlisting}

Sin embargo, el programador no se asegura que el tamaño de los datos apuntados por el string entren en el buffer local y copia ciegamente los datos, con la
función potencialmente peligrosa \textit{strcpy()}. Esto puede tranquilamente resultar en una condición de \textbf{buffer overflow}
si un atacante puede influenciar los contenidos del parámetro string. \\

\noindent \textbf{Ejemplo en C}\\

El fragmento siguiente llama a la función \textit{gets()} en C, que es inherentemente insegura.

\begin{lstlisting}[frame=single]
    char buf[24];
    printf("Please enter your name and press <Enter>\n");
    gets(buf);
\end{lstlisting}

El programador intenta copiar ciegamente desde \texttt{STDIN} al buffer sin restringir
cuanto se copia. Esto permite al usuario proveer una string que sea más larga que el tamaño del buffer, resultando en una condición de overflow.\\

\noindent \textbf{Ejemplo en C++}\\

En el siguiente ejemplo, escrito en C++, un servidor acepta conexiones de un cliente y procesa su pedido. Después de aceptar la conexión, el programa obtendrá la información del cliente
usando el método \textit{gethostbyaddr()}, copiará el hostname del cliente conectado a una variable local y también lo imprimirá en un archivo de log.

\begin{lstlisting}[frame=single]
    struct hostent *clienthp;
    char hostname[MAX_LEN];

    // create server socket, bind to server address and
    // listen on socket
    ...

    // accept client connections and process requests
    int count = 0;
    for (count = 0; count < MAX_CONNECTIONS; count++) {

        int clientlen = sizeof(struct sockaddr_in);
        int clientsocket = accept(serversocket, (struct sockaddr *)&clientaddr, &clientlen);

        if (clientsocket >= 0) {
            clienthp = gethostbyaddr((char*) &clientaddr.sin_addr.s_addr, sizeof(clientaddr.sin_addr.s_addr), AF_INET);
            strcpy(hostname, clienthp->h_name);
            logOutput("Accepted client connection from host ", hostname);

            // process client request
            ...
            close(clientsocket);
        }
    }
    close(serversocket);
    ...
\end{lstlisting}

Pero el hostname del cliente conectado podría ser más largo que el tamaño allocado para la variable local. En ese caso, ocurriría un buffer overflow cuando se copie
el hostname usando el método \textit{strcpt()}.

\subsubsection{Métodos de detección}

\begin{itemize}
    \item \textbf{Análisis Estático Automatizado}\\
        Esta vulnerabilidad puede detectarse muy a menudo utilizando herramientas de análisis estático automatizado. Muchas herramientas modernas usan análisis de flujo de datos
        o técnicas basadas en \textit{constraints} para minimizar el número de falsos positivos. En general, no toma en cuenta consideraciones ambientales al reportar overflows. Esto puede
        hacer que los usuarios encuentren difícil determinar cual \textit{warning} investigar primero. Por ejemplo, una herramienta podría reportar buffer overflows originados de argumentos
        de línea de comandos en un programa que no se supone que corra con \textit{setuid} u otros privilegios especiales.
        Tiene una alta efectividad. Las técnicas de detección para errores relacionados con buffer overflows están más maduras que para otras vulnerabilidades.
    \item \textbf{Análisis Dinámico Automatizado}\\
        Esta vulnerabilidad puede ser detectada usando herramientas y técnicas dinámicas que interactúan con el software usando largos tests con muchas y variadas entradas, como
        fuzz testing (\textit{fuzzing}), \textit{robustness testing}, y \textit{fault injection}. Se supone que el software puede ralentizarse, pero no debería volverse inestable,
        \textit{crashear}, o generar resultados incorrectos.
    \item \textbf{Análisis Manual}\\
        El análisis manual puede ser útil para encontrar esta vulnerabilidad, pero puede no lograr la cobertura de código deseada dentro los límites de tiempo deseados. Esto se torna
        complicado para vulnerabilidades donde se deben considerar todas las entradas.
\end{itemize}

\subsubsection{Nivel de vulnerabilidad}

Los niveles de vulnerabilidad pueden ser

\begin{itemize}
    \item \textit{resultante}, lo que significa que la vulnerabilidad está típicamente relacionada con la presencia de alguna otra vulnerabilidad.
    \item \textit{primario}, lo que significa que la vulnerabilidad existe independientemente de la existencia de otras vulnerabilidades.
\end{itemize}

\subsubsection{Consecuencias más frecuentes}

\begin{itemize}
    \item Integridad, Confidencialidad, Disponibilidad\\
        Los buffer overflows pueden ser usados para ejecutar código arbitrario, lo que generalmente está por fuera de las políticas de seguridad de cualquier programa,
     y lo cual puede resultar catastrófico. El impacto técnico es entonces, la ejecución no autorizada de código o comandos. Esto representa una falla de integridad,
    confidencialidad y disponibilidad.
    \item Disponibilidad \\
        Los buffer overflows generalmente llevan a \textit{crashes}. Otros ataques que llevan a la falta de disponibilidad son posibles incluyendo poner al programa en un loop infinito.
    El impacto técnico es variado: puede ser un Denial Of Service (DoS), crash, exit, restart, consumo de recursos (CPU). Esto representa una falla de disponibilidad.
\end{itemize}

\subsubsection{Formas de mitigar el error}

Existen múltiples maneras de prevenir este error. A continuación se presenta una lista con varias de estas opciones.

\begin{itemize}
    \item Elegir cautelosamente el lenguaje de programación. Esta estrategia debe ser tomada en la fase de análisis de requerimientos del sistema.
	Basta con elegir un lenguaje de programación que no permita este error o que facilite maneras de evitarlo.
	Por ejemplo, varios lenguajes que tienen un propio manejo de la memoria, como Java o Perl, no causan overflows.
    \item Usar librerías o frameworks.
	El uso de librerías o frameworks que no permitan o ayuden a prevenir este error también es de buena práctica.
	Un buen ejemplo es la \textit{Safe C String Library (SafeStr)}.
	Esto debe ser tenido en cuenta en la fase de arquitectura y diseño.
    \item Restricciones en la compilación.
	Una alternativa es correr o compilar el software usando prestaciones o servicios que tengan un mecanismo automático para prevenir este problema.
	Esta herramienta, usada en el momento de compilar no es una solución completa ya que no detecta todos los tipos de overflow, dejando posible las posibilidades de un ataque.
    \item Buenas prácticas y costumbres.
	Como siempre a la hora de la implementación resulta útil seguir una guía de buenas prácticas y costumbres.
	De entre ellas se pueden destacar:

        \begin{itemize}
            \item Volver a chequear (\textit{double-check}) si nuestro buffer tiene el tamaño que deseamos.

            \item Al copiar buffers tener en cuenta que el buffer al que se copia sea por lo menos tan grande como el original.

            \item Al estar en un ciclo, chequear que los límites del buffer estén siendo respetados, sin escribir en áreas no permitidas.

            \item Si es necesario, truncar strings a longitudes razonables.
        \end{itemize}

    \item Validar las entradas.
	Considerar que toda entrada ajena es maliciosa es una solución al problema, haciendo que solamente sean aceptables las entradas que estrictamente cumplan ciertas especificaciones.
	Es necesario considerar en este caso todas las propiedades potencialmente relevantes, incluyendo tamaño y tipo de la entrada, entre otras.
	Si bien las \textit{listas negras} pueden ser útiles para detectar potenciales ataques, no son del todo confiables, permitiendo en ciertos casos que alguna entrada mal formada o maliciosa
	pase los requerimientos.
    \item Chequeos desde el lado del servidor. Por cada chequeo de seguridad que se haga desde el lado del cliente, es una buena práctica volver a chequearlo desde el lado del servidor para
	evitar problemas.
    \item Reforzar el entorno. Usar alguna característica como \textit{Address Space Layout Randomization (ASLR)} ayuda a prevenir el problema.
	Otra forma de reforzar el entorno implica usar una computadora y un sistema operativo que ofrescan capacidades como \textit{Data Execution Protection (NX)}.
	Estas herramientas tampoco son una solución completa ya que no solucionan todos los posibles ataques.
    \item Usar funciones que incluyan parámetros de longitud. Una buena práctica es sustituir aquellas funciones sin límites de longitud, por otras que sí lo reciban por parámetro.

    \item Reducir los permisos o crear entornos reducidos. Al operar con los permisos correctos se puede lograr minimizar un ataque efectivo. Un efecto similar se obtiene al crear un
	entorno de operación con límites estrictos.
\end{itemize}

\subsubsection{Ejemplos observados}

\begin{tabular}[\baselineskip]{|p{1.75cm}|p{3.5cm}|p{8cm}|}
  \hline
  \textbf{Referencia} & Programa/Aplicación & Resumen de la vulnerabilidad \\
  \hline
  \textbf{CVE-2000- 1094} & AOL Instant Messenger(prev 4.3.2229) & Buffer Overflow por usar comando con argumento largo. \\
  \hline
  \textbf{CVE-1999- 0046} & rlogin & Buffer Overflow por usar una variable de entorno larga. \\
  \hline
  \textbf{CVE-2002- 1337} & Sendmail(5.79 a 8.12.7) & Buffer Overflow por mal parseo de comentarios. \\
  \hline
  \textbf{CVE-2003- 0595} & WiTango Application Server and Tango 2000 & Buffer Overflow, por reemplazo de un valor de una cookie por una string extramadamente larga.\\
  \hline
  \textbf{CVE-2001- 0191} & gnuserv (XEmacs) & Buffer Overflow, por reemplazo de un valor de una cookie por una string extramadamente larga. \\
  \hline
\end{tabular}


\subsection{Uncontrolled Format String}

Esta vulnerabilidad aparece cuando se le da la posibilidad al usuario de controlar el formato de impresión.
Este error se de normalmente en los sistemas con "logeo" o "internalización y localización", lo que no quiere decir que no pueda pasar en un sistema que no los tenga.
Solo se da en lenguajes que soportan texto formateado (Ej: C, C++, Pearl).


\subsubsection{Consecuencias}

Un programa afectado por esta vulnerabilidad podria ser manipulado para lograr los siguientes efectos:

\begin{itemize}

	\item Problemas de representación de datos

	\item Ver los contenidos del "stack"

	\item Escrivir al "stack"
\end{itemize}
    


\subsubsection{Detección}
Prestar atencion a los "warnings" del compilador y linkeditor
    
    
\subsubsection{Protección}
	
\begin{itemize}

	\item Elegir un lenguage que no sea suceptible a este error.

	\item Asegurarse que todos los "Strings" de formato sean estaticos
	
\end{itemize}


\subsection{Missing Encryption of Sensitive Data}
  
Esta vulnerabilidad sucede en aquellos casos en donde se envían datos sensibles sin haberlos encriptado a través de un canal no seguro.
Al omitir las garantías de seguridad e integridad no solo se le permite a un posible atacante tomar información de la comunicación, 
sino también la inyección de nuevos datos, sin permitirle a ninguna de las dos partes información válida de la inválida.

\subsubsection{Detalles técnicos del error}
\begin{tabular}[\baselineskip]{|l|p{7cm}|}
  \hline
  \textbf{Categoría} & Defensas porosas, problemas criptográficos. \\
  \hline
  \textbf{Plataforma} & Independiente. \\
  \hline
  \textbf{Lenguaje} & Independiente. \\
  \hline
  \textbf{Probabilidad de \emph{exploit}} & Alto a Muy Alto. \\
  \hline
\end{tabular}

\subsubsection{Consecuencias}

Los ataques basados en falta de encripción de datos sensibles pueden violar la confidencialidad y la integridad del sistema.

\begin{itemize}
 \item Confidencialidad: La aplicación no utiliza un canal seguro, como SSL/TLS, para  el intercambio información sensible.
	De este modo, un atacante puede acceder al tráfico de información tomando paquetes de la conexión y evaluarlos.
 \item Integridad: Al omitir la encripción de datos en cualquier aplicación puede ser considerado equivalente a enviarle la información a todos los miembros de las redes localales del
	emisor y el destinatario. Más aun, esta omisión permite que se agregue o modifique información haciendo para los usuarios imposible distinguir datos válidos de datos inválidos.
\end{itemize}

\subsubsection{Ejemplos}

\noindent \textbf{Ejemplo en PHP}\\

El siguiente código, escrito en PHP muestra este tipo de vulnerabilidad.
En este caso se escribe la información de ingreso de un usuario en una \textit{cookie} para que el usuario no tenga que volver a intgresar.
\begin{lstlisting}[frame=single]
function persistLogin($username, $password){
    $data = array("username" => $username, "password"=> $password);
    setcookie ("userdata", $data);
}
\end{lstlisting}

La información guardada en la \textit{cookie} en la computadora del usuario se encuentra como texto plano.
Esto expone las credenciales del usuario si su equipo resulta atacado. \\

\noindent \textbf{Ejemplo en Java}\\

El siguiente ejemplo, en Java, intenta establecer una conexión con un sitio para intercambiar información sensible.

\begin{lstlisting}[frame=single]
try {
	URL u = new URL("http://www.secret.example.org/");
	HttpURLConnection hu = (HttpURLConnection) u.openConnection();
	hu.setRequestMethod("PUT");
	hu.connect();
	OutputStream os = hu.getOutputStream();
	hu.disconnect();
}
catch (IOException e) {
//...
}
\end{lstlisting}

Aunque se realiza con éxito una conexión, esta conexión no esta encriptada y es posible que toda la información enviada o recibida desde el servidor sea accedida por posibles atacantes.

\noindent \textbf{Ejemplo en C}\\

El siguiente ejemplo, en C, intenta establecer una conexión, leer una contraseña y luego guardarla en un \textit{buffer}.

\begin{lstlisting}[frame=single]
server.sin_family = AF_INET; hp = gethostbyname(argv[1]);
if (hp==NULL) error("Unknown host");
    memcpy( (char *)&server.sin_addr,(char *)hp->h_addr,hp->h_length);
if (argc < 3) port = 80;
else port = (unsigned short)atoi(argv[3]);
server.sin_port = htons(port);
if (connect(sock, (struct sockaddr *)&server, sizeof server) < 0) error("Connecting");
//...
while ((n=read(sock,buffer,BUFSIZE-1))!=-1) {
	write(dfd,password_buffer,n);
	//...
}
\end{lstlisting}

Mientras resulte exitoso, este programa no encripta la información antes de escribirla en el \textit{buffer} haciendo posible la exposición de esta información.

\subsubsection{Detección}
\begin{itemize}
 \item \textbf{Análisis manual}
	Definir que información es sensible y cuál no requiere un alto conocimiento del sistema, con lo cuál un análisis manual suele ser requerido para una solución efectiva. 
	Sin embargo, un esfuerzo manual no asegura una solución total respecto a los posibles límites de tiempo. 
	Los métodos de caja negra pueden producir artefactos que necesariamente requieran análisis manual.
	Este método tiene una alta efectividad.
  \item \textbf{Análisis automatizado}
  Un análisis automatizado puede alertar la falta de encripción para ciertos datos.
  Sin embargo un análisis humano sigue siendo requerido para distinguir información que intencionalmente no esta siendo encriptada.
\end{itemize}

\subsubsection{Protección}

En una primera instancia es necesario hacer un análisis profundo de los requerimientos de confidencialidad e integridad del sistema. 
Una vez pasada esta etapa basta con encriptar correctamente los datos considerados sensibles.
Por último, el uso de un canal seguro como SSL es una buena práctica que ayuda a eliminar este problema.


\subsection{Path Traversal}

\subsubsection{Descripción del error}

Esta vulnerabilidad se da cuando un programa hace uso de entrada externa para acceder a recursos del \textit{file system}.
Si no se valida y limita correctamente a qué recursos se puede acceder, un atacante puede obtener información privilegiada o vulnerar el programa y/o sistema en el que corre.

Todo sistema que haga manejo de recursos del \textit{file system} se encuentra potencialmente afectado por esta vulnerabilidad.
No depende del sistema operativo ni del lenguaje utilizado.
Algunos lenguajes pueden ofrecer mecanismos para mitigar este ataque, pero deben ser correctamente configurados para esto.

\subsubsection{Detalles técnicos del error}
\begin{tabular}[\baselineskip]{|l|p{7cm}|}
  \hline
  \textbf{Categoría} & Manejo Riesgoso de Recursos \\
  \hline
  \textbf{Plataforma} & Independiente. \\
  \hline
  \textbf{Tiempo de Introducción} &  Arquitectura y Diseño. Implementación. \\
  \hline
  \textbf{Lenguaje} & Independiente.\\
  \hline
  \textbf{Probabilidad de \emph{exploit}} & Alta a Muy Alta \\
  \hline
\end{tabular}

\subsubsection{Ejemplos de código}

\noindent \textbf{Ejemplo en PHP}\\

\begin{lstlisting}[frame=single]
<?php
include($_GET['file']);
?>
\end{lstlisting}

Si este script se encuentra en un sistema \textit{Linux} podemos obtener la lista de usuarios del sistema y sus contraseñas encriptadas haciendo un request a \\
\texttt{/vulnerable.php?file=/etc/passwd}.

\subsubsection{Métodos de detección}

En lenguajes donde estén disponibles, herramientas de análisis estático pueden fácilmente detectar potenciales casos de \textit{Path Traversal}. Es posible que se encuentre con falsos positivos, donde la funcionalidad sea deseada para administradores o usuarios con suficientes privilegios.
En caso de no disponer de herramientas de análisis estático, el uso de \textit{White Box testing} puede detectar ocurrencias de esta clase de vulnerabilidad.

\subsubsection{Nivel de vulnerabilidad}

Los niveles de vulnerabilidad pueden ser

\begin{itemize}
    \item \textit{primario}, lo que significa que la vulnerabilidad existe independientemente de la existencia de otras vulnerabilidades.
    \item \textit{resultante}, lo que significa que la vulnerabilidad está típicamente relacionada con la presencia de alguna otra vulnerabilidad.
\end{itemize}


\subsubsection{Consecuencias más frecuentes}

Un programa afectado por esta vulnerabilidad podría ser manipulado para lograr los siguientes efectos:

\begin{itemize}

    \item Modificar archivos críticos del sistema o programa para alterar el comportamiento del mismo o otros programas en el sistema.

    \item Otorgarle al atacante acceso a partes seguras del sistema alterando archivos de seguridad.

    \item El atacante podría obtener acceso de lectura a archivos con información sensible, potencialmente otorgándole información de otros usuarios del sistema o información de seguridad del mismo.

\end{itemize}

\subsubsection{Formas de mitigar el error}

Existen numerosas maneras de proteger un programa de este tipo de error.
Como en toda vulnerabilidad de entrada externa maliciosa, se debe tratar como potencialmente hostil toda entrada, incluso si viene de una fuente confiada.

Se puede utilizar técnicas de \textit{White listing} y \textit{Black listing}.
Es decir comparar la entrada contra una lista de valores conocidos que sean considerados seguros o no.
En este caso, el uso de \textit{Black listing} no alcanza para protegerse, ya que es posible que la lista no contenga todas las posibles combinaciones maliciosas.

Ejecutar el programa dentro de un \textit{sandbox} puede ofrecer protección limitada.
Pero no ofrece protección a los recursos dentro del alcance del programa.
Un atacante podría obtener acceso privilegiado al programa, y usar este acceso para forzar al programa a salir del \textit{sandbox}, obteniendo acceso al sistema.

La mejor forma de evitar este tipo de problema es hacer uso de funciones de \textit{path canonicalization}.
Si se resuelve el \textit{path} resultante de la entrada del usuario, se puede verificar correctamente usando \textit{white listing} que no se intente manipular recursos fuera de los permitidos.

\subsubsection{Ejemplos observados}

\begin{tabular}[\baselineskip]{|p{1.75cm}|p{3.5cm}|p{8cm}|}
  \hline
  \textbf{Referencia} & Programa/Aplicación & Resumen de la vulnerabilidad \\
  \hline
  \textbf{CVE-2009-4194} & Golden FTP Server 4.30 & Un servidor FTP permite el borrado de archivos arbitrarios usando ``..'' con en el comando \texttt{DELE}. \\
  \hline
  \textbf{CVE-2009-4053} & Home FTP Server 1.10.1.139 & Un servidor FTP permite la creación de directorios arbitrarios usando ``..'' con el comando \texttt{MKD}.\\
  \hline
  \textbf{CVE-2010-0012} & Transmission 1.22 & Sobreescritura de archivos mediante el
  uso de ``..'' en un archivo torrent. \\
  \hline
  \textbf{CVE-2009-4581} & RoseOnlineCMS 3 B1 & Programa PHP permite la ejecución de código arbitrario usando ``..'' a archivos suministrados a la función \texttt{include()}. \\
  \hline
\end{tabular}


\subsection{Use of Potentially Dangerous Function}

\subsubsection{Descripción del error}

Esta vulnerabilidad consiste en la invocación de una función potencialmente peligrosa, que podría introducir una vulnerabilidad si se usa incorrectamente, 
con algún conjunto de parámetros, pero que también puede ser invocada de forma segura.

\subsubsection{Detalles técnicos del error}
\begin{tabular}[\baselineskip]{|l|p{7cm}|}
  \hline
  \textbf{Categoría} & Manejo Riesgoso de Recursos \\
  \hline
  \textbf{Plataforma} & Lenguajes de Programación \\
  \hline
  \textbf{Tiempo de Introducción} & Arquitectura y Diseño. Implementación. \\
  \hline
  \textbf{Lenguaje} & C, C++ \\
  \hline
  \textbf{Probabilidad de \emph{exploit}} & Alta \\
  \hline
\end{tabular}

\subsubsection{Ejemplos de código}
\noindent \textbf{Ejemplo en C}\\

\begin{lstlisting}[frame=single]
    void manipulate_string(char * string){
        char buf[24];
        strcpy(buf, string);
        ...
    }
\end{lstlisting}

El fragmento anterior intenta crear una copia local de un \textit{buffer} para manipular sus datos, pero no se asegura que los datos apuntados por el parámetro \textit{string} entren en el
\textit{buffer} local y se copia ciegamente los datos, con la función potencialmente peligrosa \textit{strcpy()}. Esto puede tranquilamente resultar en una condición de \textbf{buffer overflow}
si un atacante puede influenciar los contenidos del parámetro string.
En este caso, se podría evitar muy fácilmente haciendo una verificación del tamaño del parámetro string, o utilizando correctamente la función segura \textit{strncpy()}.

\subsubsection{Métodos de detección}
La detección de este tipo de funciones resulta muy díficil, principalmente porque depende de un profundo conocimiento del programador que usa la función potencialmente peligrosa.

\begin{itemize}
    \item Investigar y entender completamente el funcionamiento de una función, previo a su uso en un entorno de producción.
    \item Utilizar herramientas de análisis automático de código o compiladores que detecten el uso de este tipo de funciones.
    \item Mantenerse actualizado sobre las noticias y los parches de seguridad acerca del lenguaje o \textit{framework} que se utiliza a diario.
\end{itemize}

\subsubsection{Nivel de vulnerabilidad}

El nivel de vulnerabilidad es \textit{primario}, lo que significa que esta vulnerabilidad existe independientemente de la existencia de otras vulnerabilidades.


\subsubsection{Consecuencias más frecuentes}

Si la función se utiliza incorrectamente, es decir se le pasan ciertos parámetros que generan el comportamiento peligroso, entonces se podrían generar problemas de seguridad.
Por ejemplo,

\begin{itemize}
    \item Ejecución de código arbitrario por parte de un atacante.
    \item Que usuarios locales comunes, o atacantes remotos ganen privilegios que no deberían tener.
\end{itemize}

\subsubsection{Formas de mitigar el error}
\begin{itemize}
   \item Identificar una lista de funciones de APIs prohibidas, y prohibir su uso a los programadores, proveyendo alternativas seguras.
   \item Utilizar o configurar herramientas de análisis automático de código o compiladores para detectar el uso de estas funciones. (Ej: \textit{banned.h} de SDL de Microsoft).
\end{itemize}

\subsubsection{Extras}

Esta vulnerabilidad es diferente de la CWE-242 (Use of Inherently Dangerous Function), ya que esta última abarca funciones con tales problemas de seguridad que nunca podría ser
considerada segura. Algunas funciones, si son usadas correctamente, no suponen directamente un riesgo de seguridad, pero puede introducir una debilidad si no son
invocadas correctamente. Estas son las llamadas funciones potencialmente peligrosas. Un ejemplo muy conocido es la función \textit{strcpy()}. Cuando se le provee un buffer
de destino más largo que la fuente, \textit{strcpy()} no causará un \textbf{buffer overflow}. Sin embargo, se utiliza de manera errónea tan frecuentemente, que algunos programadores
prohiben por completo su uso.

\subsubsection{Ejemplos observados}

\begin{tabular}[\baselineskip]{|p{1.75cm}|p{4.75cm}|p{6.5cm}|}
  \hline
  \textbf{Referencia} & Programa/Aplicación & Resumen de la vulnerabilidad \\
  \hline
  \textbf{CVE-2007-1470} & LIBFtp 5.0 & Librería con multiples buffer overflows usando sprintf() and strcpy().\\
  \hline
  \textbf{CVE-2009-3849} & HP OpenView Network Node Manager(7.01-7.53) & Buffer overflow usando strcat(). \\
  \hline
  \textbf{CVE-2006-2114} & SWS web Server 0.1.7  & Buffer overflow usando strcpy(). \\
  \hline
  \textbf{CVE-2006-0963} & STLport 5.0.2  & Buffer overflow usando strcpy(). \\
  \hline
  \textbf{CVE-2011-0712} & Linux kernel(2.6.38-rc4-next-20110215) & Uso vulnerable de strcpy() cambiado a uso seguro de strlcpy().\\
  \hline
  \textbf{CVE-2008-5005} & IMAP Toolkit(2002-2007), Alpine 2.00, Panda IMAP & Buffer overflow usando strcpy(). \\
  \hline
\end{tabular}


\end{document}
