\subsection{Uncontrolled Format String}

\subsubsection{Descripción del error}

Esta vulnerabilidad aparece cuando se le da la posibilidad al usuario de controlar el
formato de un \textit{string}, habitualmente en funciones del estilo de \texttt{printf()}.
Este error se da normalmente en los sistemas con ``logueo'' o ``internacionalización y
localización'' lo que no quiere decir que no pueda pasar en un sistema que no los tenga.

\subsubsection{Detalles técnicos del error}
\begin{tabular}[\baselineskip]{|l|p{7cm}|}
  \hline
  \textbf{Categoría} & Manejo Riesgoso de Recursos \\
  \hline
  \textbf{Plataforma} & Independiente. \\
  \hline
  \textbf{Tiempo de Introducción} & Implementación \\
  \hline
  \textbf{Lenguaje} & Lenguajes que soportan texto formateado (Ej: C, C++, Perl). \\
  \hline
  \textbf{Probabilidad de \emph{exploit}} & Muy Alta. \\
  \hline
\end{tabular}

\subsubsection{Ejemplos de código}

\noindent \textbf{Ejemplo en C}\\

\begin{lstlisting}[frame=single]
int main(int argc, char **argv){
	char buf[128];
	...
	snprintf(buf,128,argv[1]);
}
\end{lstlisting}

Este código permite a un intruso ver el contenido del \textit{stack} y escribir al \textit{stack} mediante un argumento de línea de comandos que contiene una secuencia de formatos (Ej: \%d ,\%x ).
Con un \%x se puede leer contenido del \textit{stack}, mientras que con un \%n, un atacante puede escribir en el \textit{stack}.

\subsubsection{Métodos de detección}
\begin{itemize}
 \item \textbf{Black Box}\\
 	No es muy efectivo.
  \item \textbf{Análisis automatizado}\\
  	Hay herramientas de análisis estático que permiten localizar este tipo de vulnerabilidad.
\end{itemize}

\subsubsection{Nivel de vulnerabilidad}

El nivel de vulnerabilidad es \textit{primario}, lo que significa que la vulnerabilidad existe independientemente de la existencia de otras vulnerabilidades.

\subsubsection{Consecuencias más frecuentes}

Estos ataques pueden violar la confidencialidad, la integridad y la disponibilidad del sistema.

\begin{itemize}
 \item Confidencialidad\\
    Se puede revelar información oculta.
 \item Integridad y disponibilidad\\
    Se puede ejecutar código arbitrario.
\end{itemize}

\subsubsection{Formas de mitigar el errror}

\begin{itemize}

	\item Elegir un lenguage que no sea suceptible a este error.

	\item Asegurarse que todos los strings de formato sean estáticos.

	\item Prestar atención a los \textit{warnings} del compilador y linkeditor.

\end{itemize}

\subsubsection{Ejemplos observados}

\begin{tabular}[\baselineskip]{|p{1.75cm}|p{3.5cm}|p{8cm}|}
  \hline
  \textbf{Referencia} & Programa/Aplicación & Consecuencia de la vulnerabilidad \\
  \hline
  \textbf{CVE-2002-1825} & WASD (7.1 a 8.0).0 & Ejecución de código arbitrario. \\
  \hline
  \textbf{CVE-2001-0717} & ToolTalk & Ejecución de código arbitrario. \\
  \hline
  \textbf{CVE-2002-0573} & RPC daemon, Solaris(2.5.1 a 8) & Ejecución de código arbitrario. \\
  \hline
  \textbf{CVE-2002-1788} & nn (6.6.0 a 6.6.3)  & Ejecución de código arbitrario.\\
  \hline
  \textbf{CVE-2006-2480} &  Dia 0.94  & Denial of service (DoS) y posiblemente Ejecución de código arbitrario.\\
  \hline
\end{tabular}

