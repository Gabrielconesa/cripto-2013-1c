\subsection{Uncontrolled Format String}

Esta vulnerabilidad aparece cuando se le da la posibilidad al usuario de controlar el formato de impresión.
Este error se da normalmente en los sistemas con "logeo" o "internalización y localización", lo que no quiere decir que no pueda pasar en un sistema que no los tenga.


\subsubsection{Detalles técnicos del error}
\begin{tabular}[\baselineskip]{|l|p{7cm}|}
  \hline
  \textbf{Plataforma} & Independiente. \\
  \hline
  \textbf{Lenguaje} & Lenguajes que soportan texto formateado (Ej: C, C++, Pearl). \\
  \hline
  \textbf{Probabilidad de \emph{exploit}} & Muy Alto. \\
  \hline
\end{tabular}

\subsubsection{Consecuencias}

Estos ataques pueden violar la confidencialidad, la integridad y la disponibilidad del sistema.

\begin{itemize}
 \item Confidencialidad: Puede revelar información oculta. 
 \item Integridad y disponibilidad: Puede ejecutar codigo arbitrario.
\end{itemize}


\subsubsection{Ejemplos}

\noindent \textbf{Ejemplo en C}\\

\begin{lstlisting}[frame=single]
int main(int argc, char **argv){
	char buf[128];
	...
	snprintf(buf,128,argv[1]);
}
\end{lstlisting}
Este código permite a un intruso ver el contenido de el stack y escribir a el stack mediante un argumento de línea de 
comandos que contiene una secuencia de formatos (Ej: %d ,%x ).
Con un %x uno puede leer del stack.
Con un %n un atacante puede escrivir en el stack. 

\subsubsection{Detección}
\begin{itemize}
 \item \textbf{Black Box}
 	No es muy efectivo.
  \item \textbf{Análisis automatizado}
  	Hay herramientas de analisis estatico que permiten localizar este tipo de vulnerabilidad.
\end{itemize}


\subsubsection{Protección}
	
\begin{itemize}

	\item Elegir un lenguage que no sea suceptible a este error.

	\item Asegurarse que todos los "Strings" de formato sean estaticos
	
	\item Prestar atencion a los "warnings" del compilador y linkeditor
	
\end{itemize}
