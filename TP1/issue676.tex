\subsection{Use of Potentially Dangerous Function}

\subsubsection{Descripción del error}
Un programa invoca una función potencialmente peligrosa, que podría introducir una vulnerabilidad si se usa incorrectamente, con algún conjunto de parámetros, pero que
también puede ser invocada de forma segura.

\subsubsection{Detalles técnicos del error}
\begin{tabular}[\baselineskip]{|l|p{11cm}|}
  \hline
  \textbf{Categoría} & Manejo Riesgoso de Recursos \\
  \hline
  \textbf{Plataforma} & Lenguajes de Programación \\
  \hline
  \textbf{Lenguaje} & C, C++ \\
  \hline
  \textbf{Probabilidad de \emph{exploit}} & Alta \\
  \hline
\end{tabular} 

\subsubsection{Ejemplos de código}
A continuación se muestra un pequeño y simple ejemplo con código en Lenguaje C.

\begin{Verbatim}[frame=single]
    void manipulate_string(char * string){
        char buf[24];
        strcpy(buf, string);
        ...
    }
\end{Verbatim}

El fragmento anterior intenta crear una copia local de un buffer para manipular sus datos, pero no se asegura que los datos apuntados por string entren en el
buffer local y se copia ciegamente los datos, con la función potencialmente peligrosa \textit{strcpy()}. Esto puede tranquilamente resultar en una condición de \textbf{buffer overflow}
si un atacante puede influenciar los contenidos del parámetro string. 
En este caso, se podría evitar muy fácilmente haciendo una verificación del tamaño del parámetro string, o utilizando la función seguro \textit{strncpy()}.

\subsubsection{Métodos de detección}
La detección de este tipo de funciones resulta muy díficil, principalmente porque depende de un profundo conocimiento del desarrollador que usa la función potencialmente peligrosa.

\begin{itemize}
    \item Investigar y entender completamente el funcionamiento de una función, previo a su uso en un entorno de producción.
    \item Utilizar herramientas de análisis automático de código o compiladores que detecten el uso de este tipo de funciones.
    \item Mantenerse actualizado sobre las noticias y los parches de seguridad acerca del lenguaje o framework que se utiliza a diario.
\end{itemize}

\subsubsection{Nivel de vulnerabilidad}

El nivel de vulnerabilidad es \textit{primario}, lo que significa que esta vulnerabilidad existe independientemente de la existencia de otras vulnerabilidades.


\subsubsection{Consecuencias más frecuentes}

Si la función se utiliza incorrectamente, es decir se le pasan ciertos parámetros que generan el comportamiento peligroso, entonces se podrían generar problemas de seguridad. 
Por ejemplo,

\begin{itemize}
    \item Ejecución de código arbitrario por parte de un atacante.
    \item Que usuarios locales comunes, o atacantes remotos ganen privilegios que no deberían tener.
\end{itemize}
  
\subsubsection{Formas de mitigar el error}
\begin{itemize}
   \item Identificar una lista de funciones de APIs prohibidas, y prohibir su uso a los desarrolladores, proveyendo alternativas seguras.
   \item Utilizar o configurar herramientas de análisis automático de código o compiladores para detectar el uso de estas funciones. (Ej: \textit{banned.h} de SDL de Microsoft).
\end{itemize}

\subsubsection{Extras}

Esta vulnerabilidad es diferente de la CWE-242 (Use of Inherently Dangerous Function), ya que esta última abarca funciones tales problemas de seguridad que nunca podría ser
considerada segura. Algunas funciones, si son usadas correctamente, no suponen directamente un riesgo de seguridad, pero puede introducir una debilidad si no son
invocadas correctamente. Estas son las llamadas funciones potencialmente peligrosas. Un ejemplo muy conocido es la función \textit{strcpy()}. Cuando se le provee un buffer
de destino más largo que la fuente, \textit{strcpy()} no causará un \textbf{buffer overflow}. Sin embargo, se utilizada de manera errónea tan frecuentemente, que algunos desarrolladores
prohiben por completo su uso.

\subsubsection{Ejemplos observados}

Multiples buffer overflows en LIBFtp 5.0 permiten a los atacantes ejecutar código arbitrario pasando ciertos argumentos largos a las funciones FtpArchie, 
FtpDebugDebug, FtpOpenDir, FtpSize, or FtpChmod.\\

Multiples stack-based buffer overflows en HP OpenView Network Node Manager (OV NNM) 7.01, 7.51, y 7.53 permiten a los atacantes remotos ejecutar código arbitrario
pasando un parámetro Template a  nnmRptConfig.exe, relacionado con la función strcat; o un largo parámetro Oid a snmp.exe.\\

Buffer overflow en SWS web Server 0.1.7 permiten a los atacantes remotos ejecutar código arbitrario a través de un request muy largo.\\

Multiples buffer overflows en STLport 5.0.2 podrían permitir a usuarios locales ejecutar código arbitrario pasando varibles de entorno de locale largas a la función
strcpy llamada en c\_locale\_glibc2.c y argumentos largos a funciones no específicadas en num\_put\_float.cpp.\\
 
Multiples buffer overflows en la funcionalidad caiaq Native Instruments USB audio en el kernel de Linux previo a 2.6.38-rc4-next-20110215 podría permitir a
atacantes causar un denial of services (DoS) o posiblemente tener otros impactos no especificados pasando un nombre de USB device largo a las funciones
snd\_usb\_caiaq\_audio\_init en sound/usb/caiaq/audio.c y snd\_usb\_caiaq\_midi\_init en sound/usb/caiaq/midi.c.\\

Multiples stack-based buffer overflows en el  IMAP Toolkit 2002 a 2007c de la Universidad de Washington , el Alpine 2.00 y anteriores de la Universidad de Washington,
y Panda IMAP permiten:

\begin{itemize}
    \item que usuarios locales ganen privilegios especificando un argumento de extensión de carpeta largo en la línea de comando al programa tmail o dmail;
    \item que atacantes remotos ejecutarán código arbitrario enviando e-mails a una casilla de correo destino con el nombre compuesto por el nombre de usuario y un caracter '+'
    seguido de una cadena larga, procesados por el programa tmail o posiblemente dmail.
\end{itemize}