\documentclass[a4paper,10pt]{article}

\usepackage[utf8]{inputenc}
\usepackage{t1enc}

\usepackage[utf8]{inputenc}
\usepackage{t1enc}
\usepackage[spanish]{babel}
\usepackage[pdftex,usenames,dvipsnames]{color}
\usepackage[pdftex]{graphicx}
\usepackage{enumerate}
\usepackage{amsmath}
\usepackage{amsfonts}
\usepackage{amssymb}
\usepackage[table]{xcolor}
\usepackage[small,bf]{caption}
\usepackage{float}
\usepackage{subfig}
\usepackage{listings}
\usepackage{bm}
\usepackage{times}
\usepackage{verbatim}
\usepackage{moreverb}
\usepackage{fancyvrb}
% \usepackage{hyperref}
\usepackage{multirow}
\usepackage{url}
\usepackage{listings}
\lstset{breaklines=true}
\lstset{numbers=left, numberstyle=\scriptsize\ttfamily, numbersep=10pt, captionpos=b} 
\lstset{basicstyle=\small\ttfamily}
\lstset{framesep=4pt}

\begin{document}
\setcounter{secnumdepth}{5}
\setcounter{tocdepth}{5}

\begin{titlepage}
        \vfill
        \thispagestyle{empty}
        \begin{center}
                \includegraphics{./images/itba_logo.png}
                \vfill
                \Huge{Criptografia y Seguridad}\\
                \vspace{1cm}
                \Huge{Esteganografía}\\
                \vspace{1cm}
                \Huge{Trabajo Pr\'actico Especial 2}\\
        \end{center}
        \vfill
        \large{
        \begin{tabular}{lcr}
                Civile, Juan Pablo && 50453\\
                Crespo, Alvaro && 50758 \\
                Susnisky, Dario && 50592\\
        \end{tabular}
}
        \vspace{2cm}
        \begin{center}
                \large{10 de Junio de 2013}\\
        \end{center}
\end{titlepage}
\newpage

\setcounter{page}{1}

\section{Introducción}

En el presente Trabajo Práctico se muestan los resultados y análisis generados a partir del programa \textit{stegobmp}, implementado según la especificación
brindada por la cátedra. Dicho programa brinda la posibilidad de ocultar un archivo cualquiera en un archivo \textit{bmp}, mediante un método de estenografiado específico, 
con la posibilidad de encriptarlo. De igual forma, el programa permite recuperar el archivo oculto a partir de un archivo \textit{bmp}, que haya sido estenografiado con alguno 
de los métodos provistos.\\

Se presentan los resultados obtenidos a partir de las imágenes provistas por la cátedra, junto con el análisis de los mismos.
También se incluye el análisis de algunas cuestiones interesantes, establecidas por la cátedra.

\section{Desarrollo}

\subsection{Estegoanálisis de los archivos provistos}

\subsection{Cuestiones a analizar}

\subsubsection*{ 1) Para la implementación del programa \textit{stegobmp} se pide que la ocultación comience en el
primer componente del primer pixel. ¿Sería mejor empezar en otra ubicación? ¿Por qué?}

TODO

Ideas: Que hable del least significative bit, en cuyo caso conviene para que no te distorsione el color.

Sino, puede no convenir cambiar de posición igual para aprovechar todo el archivo bmp y que te permita esconder archivos mas largos.

Si reemplazamos siempre los bits menos significativos en orden, es más faćil localizar y extraer la información oculta, podría ser mejor solo ocultar 
en los bits menos significativos de una secuencia aleatoria basada en una clave o algo similar.

CITA TEXTUAL: ``normalmente esto se hace en las áreas más ruidosas de la imagen que no atrae la atención, como
por ejemplo, un prado o el cielo``.
CONVIENE  ocultar en las aŕeas más ruidosas. OK, como localizarlas?\\
''Use a pixel selection filter to obtain the best areas
to hide information in the cover image to obtain
a better rate.''



\subsubsection*{ 2) ¿Qué ventajas podría tener ocultar siempre en una misma componente? Por ejemplo, siempre
en el bit menos significativo de la componente azul.}

TODO

Es algo parecido a lo que se propone en el paper Enhanced Least Significant Bit algorithm For Image Steganography de 
Shilpa Gupta, Geeta Gujral and Neha Aggarwal. Se propone el “Enhanced Least Significant Bit (ELSB)”. 

Se te distorsiona menos la imagen...Queda mas oculto, solo se cambian los azules...
El salto de colores es menor, aunque se cambian más cantidad de pixeles.
Desventaja: requiere mayor tamaño de la imagen portadora. Para contrarrestar esto se podría hacer la modificación de solo
cambiar el bit menos significativo, sino 2, 3 o 4, que igualmente generarían una menor distorsión y requieren menor tamaño. 
(De hecho con usar 3 bits se requiere el mismo tamaño que al usar LSB-1, lo cual tiene sentido, es lo mismo, en términos de cantidades, tomar 1 bit de cada componente que 
tomar 3 de una sola componente).

\subsubsection*{ 3) Esteganografiar un mismo archivo en un .bmp con cada uno de los tres algoritmos, y comparar
los resultados obtenidos. Hacer un cuadro comparativo de los tres algoritmos estableciendo ventajas y desventajas.}

TODO
Bueno, claramente cuantos mas bits modificas mas se va l carajo la imagen. Ver bien...

\begin{tabular}[\baselineskip]{|l|p{7cm}|p{7cm}|p{7cm}}
    \hline
    Algoritmo & Ventajas & Desventajas \\
    \hline
    LSB-1 &  Menos distorsión & Requiere mayor tamaño del portador \\
    \hline
    LSB-4 & Reduce el tamaño requerido del portador & Más distorsión \\
    \hline
    LSB Enhanced &  & \\
    \hline
\end{tabular} 


\subsubsection*{ 4) Para la implementación del programa \textit{stegobmp} se pide que la extensión del archivo se oculte
después del contenido completo del archivo. ¿por qué no conviene ponerla al comienzo, después del tamaño de archivo?}

TODO

\subsubsection*{ 5) Explicar detalladamente el procedimiento realizado para descubrir qué se había ocultado en
cada archivo y de qué modo.}

TODO

\subsubsection*{ 6) ¿Qué se encontró en cada archivo?}

TODO

\subsubsection*{ 7) Algunos mensajes ocultos tenían, a su vez, otros mensajes ocultos. Indica cuál era ese mensaje
y cómo se había ocultado.}

TODO

\subsubsection*{ 8) Uno de los archivos ocultos era una porción de un video, donde se ve ejemplificado una manera
de ocultar información ¿cuál fue el portador?}

TODO

\subsubsection*{ 9) ¿De qué se trató el método de estenografiado que no era LSB? ¿Es un método eficaz? ¿Por qué?}

TODO

\subsubsection*{ 10) ¿Qué mejoras o futuras extensiones harías al programa \textit{stegobmp}?}

TODO
Que la función de extracción del archivo oculto requiera el uso de la imagen original (cover image) para poder realizar correctamente la extracción.
Fuerza que aparte de conocer la clave, se debe conocer el original de la imagen.

 Direct Cosine Transformation

 Wavelet Transformation


Implementar técnicas orientadas a archivos de texto (que utilicen archivos de texto como portadores) como Line Shift Coding Protocol o Word Shift Coding Protocol (muy interesantes),
 o White Space Manipulation(SNOW ya lo hace y es open-source). O las técnicas orientadas a archivos XML, que sería relativamente simple.

\clearpage
\appendix
\section{Anexo}

\end{document}
