\documentclass[a4paper,10pt]{article}

\usepackage[utf8]{inputenc}
\usepackage{t1enc}

\usepackage[utf8]{inputenc}
\usepackage{t1enc}
\usepackage[spanish]{babel}
\usepackage[pdftex,usenames,dvipsnames]{color}
\usepackage[pdftex]{graphicx}
\usepackage{enumerate}
\usepackage{amsmath}
\usepackage{amsfonts}
\usepackage{amssymb}
\usepackage[table]{xcolor}
\usepackage[small,bf]{caption}
\usepackage{float}
\usepackage{subfig}
\usepackage{listings}
\usepackage{bm}
\usepackage{times}
\usepackage{verbatim}
\usepackage{moreverb}
\usepackage{fancyvrb}
% \usepackage{hyperref}
\usepackage{multirow}
\usepackage{url}
\usepackage{listings}
\lstset{breaklines=true}
\lstset{numbers=left, numberstyle=\scriptsize\ttfamily, numbersep=10pt, captionpos=b} 
\lstset{basicstyle=\small\ttfamily}
\lstset{framesep=4pt}

\begin{document}
\setcounter{secnumdepth}{5}
\setcounter{tocdepth}{5}

\begin{titlepage}
        \vfill
        \thispagestyle{empty}
        \begin{center}
                \includegraphics{./images/itba_logo.png}
                \vfill
                \Huge{Criptografia y Seguridad}\\
                \vspace{1cm}
                \Huge{Esteganografía}\\
                \vspace{1cm}
                \Huge{Trabajo Pr\'actico Especial 2}\\
        \end{center}
        \vfill
        \large{
        \begin{tabular}{lcr}
                Civile, Juan Pablo && 50453\\
                Crespo, Alvaro && 50758 \\
                Susnisky, Dario && 50592\\
        \end{tabular}
}
        \vspace{2cm}
        \begin{center}
                \large{10 de Junio de 2013}\\
        \end{center}
\end{titlepage}
\newpage

\setcounter{page}{1}

\section{Introducción}

En el presente Trabajo Práctico se muestan los resultados y análisis generados a partir del programa \textit{stegobmp}, implementado según la especificación
brindada por la cátedra. Dicho programa brinda la posibilidad de ocultar un archivo cualquiera en un archivo \textit{bmp}, mediante un método de estenografiado específico, 
con la posibilidad de encriptarlo. De igual forma, el programa permite recuperar el archivo oculto a partir de un archivo \textit{bmp}, que haya sido estenografiado con alguno 
de los métodos provistos.\\

Se presentan los resultados obtenidos a partir de las imágenes provistas por la cátedra, junto con el análisis de los mismos.
También se incluye el análisis de algunas cuestiones interesantes, establecidas por la cátedra.

\section{Desarrollo}

\subsection{Estegoanálisis de los archivos provistos}

\subsection{Cuestiones a analizar}

\subsubsection*{ 1) Para la implementación del programa \textit{stegobmp} se pide que la ocultación comience en el
primer componente del primer pixel. ¿Sería mejor empezar en otra ubicación? ¿Por qué?}

TODO

Ideas: Que hable del least significative bit, en cuyo caso conviene para que no te distorsione el color.

Sino, puede no convenir cambiar de posición igual para aprovechar todo el archivo bmp y que te permita esconder archivos mas largos.

\subsubsection*{ 2) ¿Qué ventajas podría tener ocultar siempre en una misma componente? Por ejemplo, siempre
en el bit menos significativo de la componente azul.}

TODO

Se te distorsiona menos la imagen...Queda mas oculto, solo se cambian los azules...

\subsubsection*{ 3) Esteganografiar un mismo archivo en un .bmp con cada uno de los tres algoritmos, y comparar
los resultados obtenidos. Hacer un cuadro comparativo de los tres algoritmos estableciendo ventajas y desventajas.}

TODO

Bueno, claramente cuantos mas bits modificas mas se va l carajo la imagen. Ver bien...

\subsubsection*{ 4) Para la implementación del programa \textit{stegobmp} se pide que la extensión del archivo se oculte
después del contenido completo del archivo. ¿por qué no conviene ponerla al comienzo, después del tamaño de archivo?}

TODO

\subsubsection*{ 5) Explicar detalladamente el procedimiento realizado para descubrir qué se había ocultado en
cada archivo y de qué modo.}

TODO

\subsubsection*{ 6) ¿Qué se encontró en cada archivo?}

TODO

\subsubsection*{ 7) Algunos mensajes ocultos tenían, a su vez, otros mensajes ocultos. Indica cuál era ese mensaje
y cómo se había ocultado.}

TODO

\subsubsection*{ 8) Uno de los archivos ocultos era una porción de un video, donde se ve ejemplificado una manera
de ocultar información ¿cuál fue el portador?}

TODO

\subsubsection*{ 9) ¿De qué se trató el método de estenografiado que no era LSB? ¿Es un método eficaz? ¿Por qué?}

TODO

\subsubsection*{ 10) ¿Qué mejoras o futuras extensiones harías al programa \textit{stegobmp}?}

TODO

\clearpage
\appendix
\section{Anexo}

\end{document}
